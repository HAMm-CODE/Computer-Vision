% !TEX TS-program = pdflatex
% !TEX encoding = UTF-8 Unicode

% This is a simple template for a LaTeX document using the "article" class.
% See "book", "report", "letter" for other types of document.

\documentclass[12pt]{article} % use larger type; default would be 10pt

\usepackage[utf8]{inputenc} % set input encoding (not needed with XeLaTeX)

%%% Examples of Article customizations
% These packages are optional, depending whether you want the features they provide.
% See the LaTeX Companion or other references for full information.

%%% PAGE DIMENSIONS
\usepackage{geometry} % to change the page dimensions
\geometry{a4paper} % or letterpaper (US) or a5paper or....
% \geometry{margin=2in} % for example, change the margins to 2 inches all round
% \geometry{landscape} % set up the page for landscape
%   read geometry.pdf for detailed page layout information

\usepackage{graphicx} % support the \includegraphics command and options

% \usepackage[parfill]{parskip} % Activate to begin paragraphs with an empty line rather than an indent

%%% PACKAGES
\usepackage{booktabs} % for much better looking tables
\usepackage{array} % for better arrays (eg matrices) in maths
\usepackage{paralist} % very flexible & customisable lists (eg. enumerate/itemize, etc.)
\usepackage{verbatim} % adds environment for commenting out blocks of text & for better verbatim
\usepackage{subfig} % make it possible to include more than one captioned figure/table in a single float
% These packages are all incorporated in the memoir class to one degree or another...
%\usepackage{comment}

%%% HEADERS & FOOTERS
\usepackage{fancyhdr} % This should be set AFTER setting up the page geometry
\pagestyle{fancy} % options: empty , plain , fancy
\renewcommand{\headrulewidth}{0pt} % customise the layout...
\lhead{}\chead{}\rhead{}
\lfoot{}\cfoot{\thepage}\rfoot{}

%%% SECTION TITLE APPEARANCE
\usepackage{sectsty}
%\allsectionsfont{\sffamily\mdseries\upshape} % (See the fntguide.pdf for font help)
% (This matches ConTeXt defaults)

%%% ToC (table of contents) APPEARANCE
\usepackage[nottoc,notlof,notlot]{tocbibind} % Put the bibliography in the ToC
\usepackage[titles,subfigure]{tocloft} % Alter the style of the Table of Contents
\renewcommand{\cftsecfont}{\rmfamily\mdseries\upshape}
\renewcommand{\cftsecpagefont}{\rmfamily\mdseries\upshape} % No bold!


\usepackage[T1]{fontenc}
\usepackage[font=footnotesize,labelfont=bf]{caption}
\usepackage{color}
\usepackage{graphicx}
%\usepackage{subfigure}
%\usepackage{amsmath}
\usepackage{multirow}
\usepackage{booktabs,array}
\usepackage{etoolbox}
\usepackage{import}
\usepackage{amsmath,amsthm,amssymb,amsfonts}
\usepackage{fullpage}

\newenvironment{exercise}[2][Task]{\begin{trivlist}
\item[\hskip \labelsep {\bfseries #1}\hskip \labelsep {\bfseries #2.}]}{\end{trivlist}}

\newcommand{\cv}{\mathbf{c}}
\newcommand{\xv}{\mathbf{x}}
\newcommand{\tv}{\mathbf{t}}
\newcommand{\pv}{\mathbf{p}}
\newcommand{\Km}{\mathbf{K}}
\newcommand{\Tm}{\mathbf{T}}
\newcommand{\Rm}{\mathbf{R}}
\newcommand{\Mm}{\mathbf{M}}
\newcommand{\IIm}{\mathbf{I}}
\newcommand{\Wm}{\mathbf{W}}
\newcommand{\Pm}{\mathbf{P}}
\newcommand{\zerov}{\mathbf{0}}
\DeclareMathOperator{\atan2}{atan2}
\DeclareMathOperator{\trace}{trace}

\newcommand{\Rspace}{\mathbb{R}}     %euclidean and projective spaces
\newcommand{\muv}{\mathbf{\mu}}
\newcommand{\Cov}{\mathbf{\Sigma}}
\newcommand{\Am}{\mathbf{A}}

\newcommand{\lv}{\mathbf{l}}
\newcommand{\yv}{\mathbf{y}}
\newcommand{\Cm}{\mathbf{C}}
\newcommand{\Em}{\mathbf{E}}
\newcommand{\vv}{\mathbf{v}}
\newcommand{\uv}{\mathbf{u}}

%\renewcommand{\thesection}{}% Remove section references...
%\renewcommand{\thesubsection}{\arabic{subsection}}%... from subsections
%%% END Article customizations

%%% The "real" document content comes below...

\title{DATA.ML.300 Computer Vision\\ Exercise Round 1}
%\date{\vspace{-5mm} January 2026}
%\author{The Author}
\date{} % Activate to display a given date or no dat{\tiny }e (if empty),
         % otherwise the current date is printed 

\begin{document}
\maketitle

%\section{First section}

%Your text goes here.
\noindent For these exercises, you will need Python or MATLAB. 
Submit all your answers and output figures in a PDF file. For
pen \& paper tasks, the submitted PDF file should include a screenshot of your handwritten solutions, or text converted from Word or LaTeX format. In addition, submit runnable .py
or .m files, where you have filled in your codes to the given template files. Do not copy all the code from these runnable files into the PDF. 
\newline

\noindent All submissions should be uploaded to Moodle. Exercise points will be granted after a teaching assistant
has reviewed your answers. Submissions made before the deadline can earn up to 4
points. After the deadline, no partial points will be awarded; only submissions with fully correct solutions to all tasks will receive 1 point. 
\newline

\noindent \textbf{If you're using Python, make sure you have scikit-image python library installed.}
\newline

\verb|pip install --user scikit-image|
\newline 

\begin{exercise}{1}
	Homogeneous coordinates. (Pen \& paper exercise) (1 point)
	
	\noindent The equation of a line is
	\begin{equation*}
	\xv^\top\lv=0
	\end{equation*}
	This means that if a point $\xv$ lies on the line $\lv$ the equation is satisfied.
	\newline
	
	\noindent \textit{a)} Convert the four points below (cartesian x,y-coordinates) into their corresponding homogeneous coordinate form.
	\begin{equation*}
		\begin{split}
		&x_{1}=(-1,0)\\	
		&x_{2}=(2,-1)\\	
		&x_{3}=(0,1)\\	
		&x_{4}=(2,0)\\		
		\end{split}
	\end{equation*}
	
	\noindent \textit{b)}
	The line  $\lv$ through two points $\xv$ and $\xv'$ is $\lv=\xv\times\xv'$. Use this to form 
	two lines, line $\lv_{1}$ through homogeneous points $\xv_{1}$ and $\xv_{2}$, and $\lv_{2}$ through $\xv_{3}$ and $\xv_{4}$.
	\newline

	\noindent \textit{c)}
	The intersection of two lines $\lv$ and $\lv'$ is the point $\xv=\lv\times\lv'$. Use lines $\lv_{1}$ and $\lv_{2}$ to calculate their point of intersection and convert this back into cartesian coordinates.
	\newline
	


\end{exercise}

	

%\medskip
%\newpage
\begin{exercise}{2}
Image denoising (Programming exercise) (1 point)

\noindent In this exercise, you will denoise an example image using three different filters. Read the instructions below, open the \textbf{image\_denoising} file (Matlab or Python), and follow the instructions in the comments. \textbf{Include the output plot in your PDF file, and return also your runnable version of image\_denoising.} If you implemented using MATLAB, submit also your \textbf{median\_filter.m}. 
\newline
	
\noindent \textit{a)} Gaussian filtering using horizontal and vertical 1D convolutions

From Szeliski's Book section 3.2.1:
\textit{"A more direct method is to treat the 2D kernel as a 2D matrix K and to take its singular value decomposition (SVD)"}
\begin{equation*}
\mathbf{K}=\sum_{i}\sigma_{i}\mathbf{u}_{i}\mathbf{v}_{i}
\end{equation*}
\textit{"... $\sqrt{\sigma_{0}}\mathbf{u}_{0}$ and $ \sqrt{\sigma_{0}}\mathbf{v}_{0}^\top$ provide the vertical and horizontal kernels"}

Before proceeding to implement the separated Gaussian filter, separate the simpler version below to make sure you understand the process.
\begin{equation*}
\begin{bmatrix}1 & 2 & 1 \\ 
			   2 & 4 & 2 \\ 
			   1 & 2 & 1
\end{bmatrix}
\end{equation*}
Start by calculating the singular value decomposition (SVD), e.g. in Matlab, and calculate the 1D horizontal and vertical filters as described above. The resulting 1D filters $\mathbf{h}$ and $\mathbf{v}$ should be able to reconstruct the original 2D filter by simply matrix multiplying them together.


\noindent \textit{b)} Median filtering

Section 3.3.1 of Szeliski’s book.

\noindent \textit{c)} Bilateral filtering 

Section 3.3.1 of Szeliski’s book

\iffalse
\begin{equation*}
	w(i,j,k,l)=exp(-\frac{(i-k)^2+(j-l)^2}{2\sigma^2_{d}}-\frac{||f(i,j)-f(k,l)||^2}{2\sigma^2_{r}})
\end{equation*}
\fi

	
\end{exercise}

\medskip
\begin{exercise}{3}
Hybrid images (Programming exercise) (2 points)
	
\noindent In this task you will need to construct a hybrid image that combines facial images of a wolf and a man. In addition, visualize the log magnitudes of the Fourier transforms of the original images and their low-pass and high-pass filtered versions (i.e. constituents of the hybrid image). Open the \textbf{hybrid\_images} file (Matlab or Python) and follow the instructions in the comments. \textbf{Include the output plots in your PDF file and return also your runnable version of hybrid\_images.}

\end{exercise}


\end{document}
