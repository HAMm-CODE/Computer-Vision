% !TEX TS-program = pdflatex
% !TEX encoding = UTF-8 Unicode

% This is a simple template for a LaTeX document using the "article" class.
% See "book", "report", "letter" for other types of document.

\documentclass[12pt]{article} % use larger type; default would be 10pt

\usepackage[utf8]{inputenc} % set input encoding (not needed with XeLaTeX)

%%% Examples of Article customizations
% These packages are optional, depending whether you want the features they provide.
% See the LaTeX Companion or other references for full information.

%%% PAGE DIMENSIONS
\usepackage{geometry} % to change the page dimensions
\geometry{a4paper} % or letterpaper (US) or a5paper or....
% \geometry{margin=2in} % for example, change the margins to 2 inches all round
% \geometry{landscape} % set up the page for landscape
%   read geometry.pdf for detailed page layout information

\usepackage{graphicx} % support the \includegraphics command and options

% \usepackage[parfill]{parskip} % Activate to begin paragraphs with an empty line rather than an indent

%%% PACKAGES
\usepackage{booktabs} % for much better looking tables
\usepackage{array} % for better arrays (eg matrices) in maths
\usepackage{paralist} % very flexible & customisable lists (eg. enumerate/itemize, etc.)
\usepackage{verbatim} % adds environment for commenting out blocks of text & for better verbatim
\usepackage{subfig} % make it possible to include more than one captioned figure/table in a single float
% These packages are all incorporated in the memoir class to one degree or another...

%%% HEADERS & FOOTERS
\usepackage{fancyhdr} % This should be set AFTER setting up the page geometry
\pagestyle{fancy} % options: empty , plain , fancy
\renewcommand{\headrulewidth}{0pt} % customise the layout...
\lhead{}\chead{}\rhead{}
\lfoot{}\cfoot{\thepage}\rfoot{}

%%% SECTION TITLE APPEARANCE
\usepackage{sectsty}
\allsectionsfont{\sffamily\mdseries\upshape} % (See the fntguide.pdf for font help)
% (This matches ConTeXt defaults)

%%% ToC (table of contents) APPEARANCE
\usepackage[nottoc,notlof,notlot]{tocbibind} % Put the bibliography in the ToC
\usepackage[titles,subfigure]{tocloft} % Alter the style of the Table of Contents
\renewcommand{\cftsecfont}{\rmfamily\mdseries\upshape}
\renewcommand{\cftsecpagefont}{\rmfamily\mdseries\upshape} % No bold!


\usepackage[T1]{fontenc}
\usepackage[font=footnotesize,labelfont=bf]{caption}
\usepackage{color}
\usepackage{graphicx}
%\usepackage{subfigure}
%\usepackage{amsmath}
\usepackage{multirow}
\usepackage{booktabs,array}
\usepackage{etoolbox}
\usepackage{import}
\usepackage{amsmath,amsthm,amssymb,amsfonts}
\usepackage{fullpage}
\usepackage{url}

\newenvironment{exercise}[2][Task]{\begin{trivlist}
\item[\hskip \labelsep {\bfseries #1}\hskip \labelsep {\bfseries #2.}]}{\end{trivlist}}

\newenvironment{demo}[2][Demo]{\begin{trivlist}
\item[\hskip \labelsep {\bfseries #1}\hskip \labelsep {\bfseries #2.}]}{\end{trivlist}}

\newcommand{\cv}{\mathbf{c}}
\newcommand{\xv}{\mathbf{x}}
\newcommand{\yv}{\mathbf{y}}
\newcommand{\zv}{\mathbf{z}}
\newcommand{\wv}{\mathbf{w}}
\newcommand{\logit}{\sigma}
\newcommand{\softmax}{s}

\newcommand{\tv}{\mathbf{t}}
\newcommand{\pv}{\mathbf{p}}
\newcommand{\Km}{\mathbf{K}}
\newcommand{\Tm}{\mathbf{T}}
\newcommand{\Rm}{\mathbf{R}}
\newcommand{\Mm}{\mathbf{M}}
\newcommand{\IIm}{\mathbf{I}}
\newcommand{\Wm}{\mathbf{W}}
\newcommand{\Pm}{\mathbf{P}}
\newcommand{\zerov}{\mathbf{0}}
\DeclareMathOperator{\atan2}{atan2}
\DeclareMathOperator{\trace}{trace}

%\newcommand{\redtext}[1]{\textcolor{red}{{#1}}}
%\renewcommand{\thesection}{}% Remove section references...
%\renewcommand{\thesubsection}{\arabic{subsection}}%... from subsections
%%% END Article customizations

%%% The "real" document content comes below...

\title{DATA.ML.300 Computer Vision\\ Exercise Round 2}
\date{\vspace{-5mm} January 2025}
%\author{The Author}
\date{} % Activate to display a given date or no date (if empty),
         % otherwise the current date is printed 

\begin{document}
\maketitle

%\section{First section}

%Your text goes here.
\noindent For these exercises you will need MATLAB, which should be available on university computers. Submit all your answers and output figures in a PDF file. For pen \& paper tasks, the submitted PDF file should include a
screenshot of your handwritten solutions, or text converted from Word or LaTeX format.
In addition, submit runnable .m files, where you have filled in your codes to the
given template files. Do not copy all the code from these runnable files into the PDF.
\newline

\noindent All submissions should be uploaded to Moodle. Exercise points will be granted after a
teaching assistant has reviewed your answers. Submissions made before the deadline can
earn up to 4 points. After the deadline, no partial points will be awarded; only submissions with fully correct solutions to all tasks will receive 1 point.

%\vspace{2.5mm}
%\noindent Get the example \texttt{m}-files by downloading \texttt{Exercise10.zip} from the MyCourses page. 


\begin{exercise}{1}
	Neural networks and backpropagation. (Pen \& paper problem) (1 points)
	
	\noindent	
	In Figure 1 below you see a very small neural network, which has one input
	unit, one hidden unit (logistic), and one output unit (linear). The nonlinear
	function in the logistic unit is defined by the formula $\sigma(x)=\frac{1}{1+exp(-x)}$.
	Let’s consider one training case. For that training case, the input value is 1
	(as shown in the figure) and the target output value t is 1. We are using the
	standard squared loss function: $E=\frac{1}{2}(t-y)^2$, where y is the output of the
	network. 
	\newline
	
	\noindent \textit{a)}
	What is the output of the hidden unit and the output unit, for this training
	case?
	
	\noindent \textit{b)}
	What is the loss, for this training case?
	
	\noindent \textit{c)}
	What is the derivative of the loss with respect to $w_2$, for this training
	case? \textit{Hint: Use chain rule}
	
	\noindent \textit{d)}
	What is the derivative of the loss with respect to $w_1$, for this training
	case? \textit{Hint: the derivative of logistic function is defined as $\frac{d}{dx} \sigma(x)= \sigma(x) \cdot (1- \sigma(x))$}
		
	\begin{figure}
		\begin{center}
		\includegraphics[scale=0.2]{network.png}
		\caption{A small neural network with one hidden unit. The values for the weights and
			biases are given in the figure}
		\end{center}
	\end{figure}
	\end{exercise}

\newpage
\begin{exercise}{2} Matching images based on similarity (Pen \& paper problem) (1 point)

\vspace{1mm}
\noindent We have three feature vectors which are defined as 
\begin{gather*}
    \mathbf{Q} = \begin{bmatrix}
    2 & 1 & 6 & 4 & 2
    \end{bmatrix}^T, \\
    \mathbf{A} = \begin{bmatrix}
    1 & 2 & 3 & 4 & 1
    \end{bmatrix}^T, \\
    \mathbf{B} = \begin{bmatrix}
    3 & 1 & 4 & 1 & 5
    \end{bmatrix}^T,
\end{gather*}

\noindent where $\mathbf{Q}$ is a feature vector extracted from a query image Q, and $\mathbf{A}$ and $\mathbf{B}$ are feature vectors extracted from random images A and B in a dataset.
\newline

\noindent \textit{a)}
	Calculate the Euclidean distance and cosine similarity between $\mathbf{Q}$ and $\mathbf{A}$ and between $\mathbf{Q}$ and $\mathbf{B}$.
	

\noindent \textit{b)}
Based on the feature vectors and similarity metrics calculated in a), which image from the dataset is more similar to query image Q? Why?

\end{exercise}

\begin{exercise}{3} Observing different parts of a simple CNN. (Programming exercise) (2 points)

\vspace{1mm}
\noindent Download the material and open the exercise.m file. Each section observes different building blocks of a convolutional neural network. Your task is to progress one section at a time, fill any missing code and answer questions asked in each section. \textbf{Questions can be answered in the code by inserting comments below the questions; however, if you prefer PDF, include the questions also.}

\end{exercise}





%\vspace{2.5mm}
%\noindent Tasks continue on the next page...
%\newpage
%\

\end{document}